\documentclass[a4paper,12pt]{extarticle}
\usepackage[T1]{fontenc}
\usepackage[utf8]{inputenc}
\usepackage{geometry}
\usepackage{geometry}
 \geometry{
 a4paper,
 total={170mm,247mm},
 left=20mm,
 top=30mm,
 }

% macros
\newcommand{\doubleLine}{
    \vspace*{20px} \\
}
\newcommand{\granice}[2]{
    _{#1}^{#2}
}

% end of macros


\title{%
  Wojskowa Akademia Techniczna \\~\\
  \large Modelowanie Matematyczne Ćwiczenia \\~\\
    \textbf{Prowadzący:} mgr. inż, Michał Kapałka \\~\\
    \textbf{Temat:} Problem przydziału zadań zespołowi programistów}
\author{\textbf{Autor:} Michał Konopka}
\date{Czerwiec 2020}

\begin{document}

\maketitle
\newpage

\section{Opis problemu}
Istnieje zbiór wątków które procesory muszą rozwiązać w skończonym czasie. Pewne zadania wymagają specjalnych zasobów, innym wystarczy sam procesor. Zasób może być używany tylko przez jeden proces. Wszystkie wątki powinny być wykonane w jak najkrótszym czasie z wykorzystaniem jak najmniejszej ilości energii. Każdy procesor ma inną szybkość i inną proporcję szybkość do zużycia energii. Proporcja ta zmienia się w stanie spoczynku procesora. Każdy wąek może być wykonywany tylko na jednym procesorze.
\section{Model matematyczny opisanego problemu}
    \subsection{Matematyczny opis cech istotnych}
        \begin{itemize}
            \item	$L_p$ -  liczba procesów
            \item   $L_w$ -  liczba wąków
            \item	$L_z$ -  liczba zasobów
            \item	$L_s$ -  zbiór identyfikatorów procesów
            \item   $Z$ - zbiór wszystkich identyfikatorów zadań
            \item   $A$   - zbiór identyfikatorów wszystkich programistów NOWE
            \item   $M$   - zbiór identyfikatorów wszystkich młodszych programistów NOWE
            \item   $S$   - zbiór identyfikatorów wszystkich starszych programistów NOWE
            \item   $D_e$ - Maksymalny "błąd" dyspozycyjności 
            \item	$W_{j,z}$- wydajność j-tego programisty dla z-tego zadania(mierzona w poziomach zadania na godzinne)
            \item	$P_z $– Trudność z -tego zadania wyrażana w poziomach
            \item	$D_j$ – Dyspozycyjność j – tego programisty w godzinach
            \item	$T_z$ – Czas przeznaczony na wykonanie z – tego zadania
            \item	$MP$ – Płaca młodszego programisty na godzinne
            \item	$SP$ – Płaca starszego programisty na godzinne
            \item	$T_{mz}$ – Czas przeznaczony przez m -tego młodszego programistę na z- te zadanie 
            \item	$T_{sz}$ – Czas przeznaczony przez s- tego młodszego programistę na m- te zadanie 
            \item	$T_{pm}$ – Czas przeznaczony przez p- tego  programistę na m- te zadanie 
            \item	$K$ – Koszt wykonania zbioru zadań
            \item	$T_{zsm}$ – Czas wymagany na nadzór m – tego młodszego programisty , przez s – tego starszego programisty przy z -tym zadaniu
            \item	$L_c$ – liczba cech specjalnych
            \item	$C_{ij}$ – posiadanie i -tej cechy przez j -tego programistę
            \item	$C_{iz}$ – wymaganie i -tej cechy do z -tego zadania 
        \end{itemize}
        Opis cech: $$\dot{X}=\{<L_z,N_+>,<L_j,N_+>,<L_s,N_+>,<\{P\}_{i=1}^{L_j + L_s}>$$
        $$\{\{<W_{j,z},N_+>\}_{z=1}^{L_z}\}_{j=1}^{L_j+L_s},\{<P_z,N_+>\}_{z=1}^{L_z},\{<D_j,N_+>\}_{j=1}^{L_j+L_s},$$
        $$\{T_z,N_{+\}_{z=1}^{L_z}},<MP,N_+>,<SP,N_+>,\{\{<T_{mz},N_+>\}_{m=1}^{L_j}\}_{z=1}^{L_z},\{\{<T_{sz},N_+>\}_{m=1}^{L_s}\}_{z=1}^{L_z},$$
        $$<K,R_+>,\{\{\{T_{zsm},N_+\}_{z=1}^{L_z}\}_{s=1}^{L_s}\}_{j=1}^{L_j},<L_c,N_+>,$$
        $$\{\{<C_{ij},\{0,1\}>\}_{i=1}^{L_c}\}_{j=1}^{L_s+L_j},$$
        $${\{\{<C}_{iz},{0,1}{>\}}_{i=1}^{L_c}\}_{i=1}^{L_z}\}\}$$

\section{Matematyczny opis istotnych związków między wybranymi cechami}
    \begin{itemize}
        \item $R_1$ : Czas przeznaczony przez programistę na wszystkie zadania może
            różnić się maksymalnie o 5h od jego dyspozycyjności \doubleLine
            $ R_1 = \{<\{A\}_{i=1}^{L_j + L_s}, \{\{T_{pm}\}_{p=1}^{L_j + L_s}\}_{m=1}^{L_z}, D_e> \in {?} : \forall_{p \in A} \sum_{i=1}^{L_z} T_{pi} - D_p < \pm D_e   \} $
        \item $R_2$ : Wszystkie zadania muszą zostać zrealizowane w określonym czasie
            \\~\\
            $ R_2 = \{  \}$
        
        
    \end{itemize}
\end{document}


